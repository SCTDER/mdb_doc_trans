\section{数据操作}\label{dataop}

\linespread{1.2}
\large
\subsection{说明}
本产品操作对象为数蚕内存数据对象,数蚕内存数据库基本对象为数据表, 多个数据表组成一个库。

表中使用列式存储字段,因此数据表结构对象为列(或称作字段)。

每个对象提供常规的增删改查操作。

\subsection{数据库操作}
\bigskip
本产品针对内存数据库提供数据库服务端\textbf{重连}、\textbf{刷新}、\textbf{创建表}、\textbf{加载表}、\textbf{备份}、\textbf{还原}及\textbf{执行sql}文件功能。

执行数据库操作首先在项目窗口选择根数据库节点,点击右键或对应数据库菜单或工具栏对应按钮。

\textbf{重连}用于判定服务端是否工作正常,重连成功后系统状态栏左下角状态图标为连接状态否则显示断开状态。

\textbf{刷新}用于展示左侧节点树, 默认创建,修改表结构左侧不更新节点状态,使用刷新显示当前服务端数据库结构状态。

\textbf{创建表}用于新建一个内存表,新建表需要指定表名并至少创建一个字段,每个字段需要指定字段名称、字段类型、及是否可重复。
使用表名右侧+/-可以添加或删除当前选择字段。点击应用后创建对应表,日志窗口输出执行语句,结果及时间。
相应表名,字段名及更多规范请参考\href{http://www.shucantech.com/zh/sql_ref_0.2.0.html}{在线数蚕sql规范}。

\textbf{加载表}功能是直接加载二进制或sql文件为一个内存表。
二进制表和sql文件区另是二进制保存和加载更快速和体积更小,sql文件移植性和可纠错性更好。您可以根据你的使用需求选择文件格式。
加载表可以通过服务端加载也可以传送客户端表文件,加载客户端一次读取所有数据因此可能不太适用特别大的数据文件,建议针对大文件使用服务端加载。
加载表可以指定二进制表文件和sql表文件,二进制表文件不含有表名信息,因此需要您指定导入的表名。sql文件中使用create语句构造,无需指定表名。
使用服务端请确保保存时服务端存在输入的表文件。否则可能加载数据表失败。
无论哪种方式加载新表请确保相应表不存在于库中,否则加载表将导致失败。

\textbf{备份}功能是备份整体数据库为单一数据文件。
数据库根据您的选择备份所有表为二进制或sql文件。替换模式时会直接替换已经存在的文件。
相应服务端和客户端同加载表功能类似,建议针对大文件使用服务端加载。

\textbf{还原}功能是将整体数据库从单一数据文件还原。
数据库根据您的选择从原始备份中还原整个数据库,注意还原数据库会还原所有数据表到目标文件状态,还原时请注意数据备份。
相应服务端和客户端同备份时功能类似,建议针对大文件使用服务端加载。

\textbf{执行sql}文件直接执行客户端上的sql脚本文件。相应文件执行出错则给出出错位置或者提示执行成功信息。

\subsection{数据表操作}
\bigskip

选择左侧表节点可以对数据库进行表操作。

本产品对于表示操作提供\textbf{删除表}、\textbf{清空表}、\textbf{查看字段}、\textbf{添加字段}、\textbf{查看或修改数据}以及整表的\textbf{导入}、\textbf{导出},\textbf{保存}操作。

\textbf{删除表}用于从内存数据库删除指定节点名的表对象。删除后您需要刷新左侧节点树查看数据库结构。

\textbf{清空表}用于快速清空内存数据表, 但仍保留表结构信息。

\textbf{查看字段}结构显示数据表的字段结构信息,包含字段名、字段类型和是否重复。

\textbf{添加字段}用于添加新的字段到表中,字段类型和\href{http://www.shucantech.com/zh/sql/sql_ref_0.2.0.html#overview}{sql规范概述}中类型名称一致。并设置为默认值。在添加字段窗口表名右侧可以使用
+/-号标识的按钮添加或删除一个字段。每行字段信息可以使用表格进行直接编辑,点击应用后生成sql并执行语句相应信息输出于日志窗口中。

\textbf{查看或修改数据}表数据显示窗口,默认窗口位于最左侧。数据查看窗口默认输出表格的前1000行数据,您可以取消表格输出限制,显示所有表格数据。
同时您可以使用过滤条件进行数据过滤,过滤条件默认为where条件内容。使用where条件过滤内容可以直接在表格中进行修改,点击应用后相应表格修改内容会重新写入内存数据表中。
后置条件可以进一步对输出数据过滤处理,相应后置条件可参考\href{http://www.shucantech.com/zh/sql/sql_ref_0.2.0.html#process}{后置处理流程}说明。
{\color{orange}注意}: 部分后置条件处理后数据表内容不可进行修改,您可能无法编辑相应的输出表格内容。

过滤和后置处理下拉框中默认添写了一些常用的过滤字段,您可以也通过对表格内容选择使用右键菜单\textbf{过滤}或\textbf{追加过滤}过滤对应表格内容。

数据窗口使用字段列表进行显示字段过滤,勾选或取消相应表字段则显示或不显示对应字段数据。
数据窗口中可以使用多种显示格式
规范化显示所有数据为数蚕sql的规范形式,这包含符号位显示,整数值十六进制显示,后缀符号显示。
十六进制显示显示所有整数值为十六进制,编辑时16位分组,分组以空格显示。编辑时可以添加也可删除分组空格。空格对目标值无影响。
二进制显示显示所有整数值为二进制,编辑时4位分组,分组以空格显示。编辑时可以添加也可删除分组空格。空格对目标值无影响。
十进制显示显示所有整数值为十进制,编辑时3位分组,分组以空格显示。编辑时可以添加也可删除分组空格。空格对目标值无影响。
修改后数据默非立即模式认显示为绿色,立即模式立即进行更新修改。

查找和替换
使用菜单查找或替换可以对当前表按约束进行查找或替换。
查找到的对象以黄色显示。
默认无范围约束,即全表范围,使用行约束只查找或替换当前行,列约束只查找或替换当前列。

查找功能支持大小写敏感查找和整词查找选项。
使用普通模式查询目标串于全文中不作特别处理。
使用正则表达式模式把目标串当作正则表达式处理,默认正则表达式使用C++17默认正则表达式参数串处理。
使用扩展模式把目标串仅把目标串中进行标准转义(C++17标准)处理。

使用\textbf{下一个}和\textbf{上一个}按钮可以查找当前位置的后一个和前一个匹配串。
查找全部显示所有匹配串。

使用替换可以对当前串进行替换,替换全部则全部替换。

表的\textbf{导入}/\textbf{导出}用于导入/导出交换的csv或csvs文件,csv或csvs文件规范请参见\href{http://www.shucantech.com/zh/sql/sql_ref_0.2.0.html#import}{导入功能}说明。
注意导入表时使用的是服务端文件路径,请确保服务端有此文件路径。根据导入文件内容选择字段分隔符号,如分隔符号不存在可以直接在下拉框中输入分隔串。
分隔符号可以为任意多个非行尾(\textbackslash n)符号。

表的\textbf{保存}功能是直接保存内存数据库为二进制文件,可用于快速的数据保存。
保存的文件总是一个完整的数据表内容,请确保保存时服务端不存在同表名相同的文件。
否则可能保存数据失败。


\subsection{数据字段操作}
\bigskip

选择左侧字段节点可以对数据库进行字段操作。

本产品对于字段操作提供\textbf{删除字符}、\textbf{重命名字段}、\textbf{修改字段}功能。

\textbf{删除字段}用于从内存数据表删除指定节点名的字段对象。删除后您需要刷新左侧节点树查看数据库表结构。

\textbf{重命名字段}可以修改目标字段名为新的字段名,其它信息不可修改。

\textbf{修改字段}用于修改字段类型和是否重复参数。字段名称不可修改,您也可以通过修改字段功能添加更多字段信息。 +/-号标识的按钮添加或删除一个字段。
每行字段信息可以使用表格进行直接编辑,点击应用后生成sql并执行语句相应信息输出于日志窗口中。
