\section{文件操作}\label{file}

\linespread{1.2}
\large

\subsection{文件创建、打开}
\bigskip
本产品使用\textbf{文件}菜单下\textbf{新建}或双击文件编辑器空白处新建的一个sql文件。默认为创建的sql生成一个未使用的文件名,新建的文件必须指定一个文件保存位置。

本产品使用\textbf{文件}菜单下\textbf{打开}打开一个保存的sql文件,如何文件编辑器中已经打开目标文件,则显示对应编辑器窗口。

打开或新建文件成功则显示于对应编辑器窗口。

您可以通过\textbf{打开最近}菜单选择打开最近使用的文件,默认保存历史文件20个,您可以使用\textbf{清除列表}子菜单清空您的打开记录。

\subsection{文件保存}
\bigskip
本产品使用\textbf{文件}菜单下\textbf{保存}保存当前编辑器sql文件。

本产品使用\textbf{文件}菜单下\textbf{保存全部}保存所有编辑器sql文件。

本产品使用\textbf{文件}菜单下\textbf{另存为}另存当前编辑器sql文件。

\subsection{文件关闭}
本产品使用\textbf{文件}菜单下\textbf{关闭}关闭当前编辑器sql文件。关闭文件时默认会进行保存操作, 然后关闭对应编辑器窗口。

本产品使用\textbf{文件}菜单下\textbf{关闭全部}关闭全部编辑器sql文件。关闭文件时默认会进行保存操作, 然后关闭所有编辑器窗口。

本产品使用\textbf{文件}菜单下\textbf{关闭其它}关闭非当前编辑器sql文件。关闭文件时默认会进行保存操作, 然后关闭非当前编辑器窗口。

\subsection{文件打印}
\bigskip
本产品使用\textbf{文件}菜单下\textbf{打印}打印当前编辑器sql文件。目前打印设置仅仅按编辑器页面大小进行。

\subsection{退出}
\bigskip
本产品使用\textbf{文件}菜单下\textbf{退出}菜单退出系统,退出时保存所有编辑器内容,并保存打开文件窗口状态。{\color{red}注意}:退出时不保存非编辑器页面内容。
