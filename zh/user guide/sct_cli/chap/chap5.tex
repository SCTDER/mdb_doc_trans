\section{编辑指导}\label{edit}

\linespread{1.2}
\large
\subsection{sql规范}
本产品中sql代码应遵守\href{http://www.shucantech.com/zh/sql_ref_0.2.0.html}{数蚕sql规范0.2.0版本}。
基本要求为编辑代码编码默认为UTF-8格式,行尾使用\textbackslash n格式。其它请参见上述在线参考。

\subsection{代码样式和主题}
\bigskip

本产品针对不同编码人员提供三种风格的主题风格。分别是\textbf{雅致}、\textbf{清新}和\textbf{静逸}。您可以通过\textbf{选项}菜单下
\textbf{首选项}中设置编辑器样式和主题。

针对不同的高亮对象您可以分别选择不同对象并设置右侧的字体和颜色样式自定义您自己的颜色主题。
通过主题右侧\textbf{重置为默认}您可以恢复默认的主题设置。

编辑器样式设置中水平分隔因子表示编辑器中每个字符之间的水平间距相对于每个字体点数比值。值越小字体之间水平距离越小。

编辑器样式设置中垂直分隔因子表示编辑器中行间距相对于每个字体点数比值。值越小行间距离越小。

编辑器显示空格可以设置显示ASCII 0x20空格字符为一个点号。

在编辑器编辑选项中可以设置显示光标的颜色。

其它样式功能目前版本可能未实现。不建议使用。

编辑器行号显示总是从1开始到编辑内容最后一行。

\subsection{代码编辑}
\bigskip

本产品sql编辑支持常用的编辑功能。
默认编辑快捷键使用Windows风格编辑快捷键。

快捷键列表如下:\\
\begin{verbatim}
        Ctrl+A            全部选择
        Ctrl+Y            重做编辑
        Ctrl+Z            撤消编辑
        Ctrl+C            复制选择
        Ctrl+X            剪切选择
        Ctrl+V            粘贴
        Ctrl+E            跳转到错误
        Ctrl+W            选择整词
        Ctrl+Left         转到词首
        Ctrl+Right        转到词尾
        Home              转到行首
        Ctrl+F            查找
        Ctrl+H            替换
        Delete            删除选择或后一个字符
        Backspace         删除选择或前一个字符

\end{verbatim}

其中Ctrl指键盘左下角控制键,Left为左方向键($\leftarrow$),Right为右方向键($\rightarrow$)。

按住Shift移动左方向键或点击鼠标按住拖动可以选择多个字符。
双击默认选择一个词,1秒内再次双击选择整行,1秒内再双击选择全部。

选择一个词默认使用黄色显示所有相同词。任意编辑行为取消显示内容。

\subsection{查找和替换}
\bigskip

 使用菜单或相应快捷键可调出查找对话框。相应功能工作于当前活动编辑窗口。

 查找功能支持大小写敏感查找和整词查找选项。
 使用普通模式查询目标串于全文中不作特别处理。
 使用正则表达式模式把目标串当作正则表达式处理,默认正则表达式使用C++17默认正则表达式参数串处理。
 使用扩展模式把目标串仅把目标串中进行标准转义(C++17标准)处理。

 使用\textbf{下一个}和\textbf{上一个}按钮可以查找当前位置的后一个和前一个匹配串。
 查找全部显示所有匹配串。

 使用替换可以对当前串进行替换,替换全部则全部替换。


\subsection{智能提示}
\bigskip

本产品可针对编辑的sql进行智能的补全提示。
您可以通过\textbf{首选项}中编辑器\textbf{自动补全}选项段中\textbf{启动}或\textbf{禁用}。
目前只要输入内容有提示则自动进行智能提示,且只补全sql语言部分及字段表名内容。

自动补全是基于sql语义的智能补全,使用最长匹配原则对输入语句进行解析处理。
智能提示不访问服务端,因此只针对基础语法语义解析。智能提示无法作为准确的补全依据。使用时仅提供参考。

智能提示可用时,弹出补全提示列表;可以通过Ctrl+J选择当前串的下一个串,Ctrl+K选择当前串的上一个串。
使用Tab/Enter或鼠标选择补全条目作为目标。如果输入补全不存在有效补全内容,可直接继续输入,相应输入内容会直接作为输入。
任意的不可见字符均作为取消匹配处理。取消时需要从原始补全处理重新输入补全内容。

\subsection{错误修改}
\bigskip

默认编辑的内容会进行语法分析,语法出现错误则把在第一个出错处显示为红色字体,后继显示为加下划线的显示效果。
您可以通过\textbf{编辑}菜单下的\textbf{转到错误}菜单转到出错位置进行错误修改。

编辑中的错误检查仅作语法检查,相应sql执行结构仍然可能失败。执行sql后您可以通过日志窗口查看相应错误信息。
更复杂的错误信息您需要分析sql及数据内容进行排查。

\subsection{保存和加载}
\bigskip

sql编辑器内容每次软件正常关闭时会进行保存, 再次打开时自动打开对应的编辑窗口。
sql内容也可以使用\textbf{文件}菜单中\textbf{保存}/\textbf{打开}文件功能。相应文件使用UTF-8编码和\textbackslash n换行符。
